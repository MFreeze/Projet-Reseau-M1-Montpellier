\documentclass[a4paper]{article}
\usepackage[francais]{babel}
\usepackage[T1]{fontenc}
\usepackage[applemac]{inputenc}
\usepackage{geometry}
\usepackage{graphicx}
\usepackage[colorlinks=true]{hyperref}
\hypersetup{urlcolor=blue,linkcolor=black,colorlinks=true} 
\usepackage{algorithm,algorithmic}
\usepackage{listings}
\usepackage{amssymb}
\usepackage{setspace}
\usepackage{listings}
\usepackage{lscape}

\pagestyle{headings}
\thispagestyle{empty}
\geometry{a4paper,twoside,left=2.5cm,right=2.5cm,marginparwidth=1.2cm,marginparsep=3mm,top=2.5cm,bottom=2.5cm}
\begin{document}
\large
\setlength{\parindent}{ 0 pt}
\lstset{language=C, showstringspaces=false, numbers=left, numberstyle=\tiny, tabsize=4}
\setlength{\parskip}{5mm plus2mm minus2mm}
Chlo� DESDOUITS \hfill M1 Informatique MOCA\\
Guillerme DUVILLIE
\vfill
{\centering \Huge \bfseries Projet de r�seau \par}
\vfill
23 d�cembre 2011 \hfill UM2

\newpage
\tableofcontents
\thispagestyle{empty}

\setlength{\parindent}{1 cm}
\pagenumbering{arabic}
\newpage


\section{Mode d'emploi}\label{manuel}
Tout d'abord, notre application n�cessite l'installation de la biblioth�que \href{http://ftp.gnu.org/pub/gnu/ncurses/}{Ncurses} qui g�re l'interface console. Il suffit ensuite de compiler gr�ce au Makefile.

L'ex�cutable du serveur se nomme "serveur" et accepte les param�tres suivants :

\begin{itemize}
\item -n nom de l'hote (par default 'localhost')
\item -i adresse ip de l'hote (par default '127.0.0.1')
\item -p numero de port qu'utilisera le serveur d'envoi (par default '13321')
\item -ps numero de port secondaire qu'utilisera le serveur de r�ception (par default '13322')
\end{itemize}


L'ex�cutable du client se nomme "client" et accepte les param�tres suivants :

\begin{itemize}
\item -n nom de l'hote (par default 'localhost')
\item -a adresse ip de l'hote (par default '127.0.0.1')
\item -p numero de port qu'utilise le serveur d'envoi (par default '13321')
\item -P numero de port qu'utilise le serveur de r�ception (par default '13322')
\end{itemize}

Les ex�cutables se quittent en envoyant le signal SIGINT (ctrl-c).


\section{Architecture de l'application}\label{archi}
Commen�ons par examiner l'architecture du serveur. Notre serveur est divis� en deux processus, les deux �tant multi-thread.

Le processus p�re est le serveur d'envoi ; il g�re le broadcast de la grille. Son thread principal accepte les connexions des clients et cr�e un thread secondaire par client connect�. Les threads secondaires envoient la grille � leur client respectif.

Le processus fils est le serveur de r�ception ; il traite les demandes de d�placement de la cam�ra. Son thread principal g�re la file d'attente des clients qui veulent d�placer la cam�ra. Il lance un thread secondaire pour le premier client qui demande la main. Ce thread secondaire g�re le d�placement du pointeur dans la grille. Le thread principal tue le thread secondaire � la fin du temps imparti et relance un thread secondaire pour le client suivant.\\


Le client, quant � lui,


\section{Protocoles d'�change}\label{protocoles}


\section{Sch�mas algorithmiques}\label{algo}


\section{Difficult�s et solutions}\label{difficultes}


\vfill
{\raggedleft R�alis� avec \LaTeX{} \par}

\end{document}
